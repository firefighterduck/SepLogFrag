\documentclass[11pt,a4paper]{article}
\pagestyle{plain} % turn on page numbers

\usepackage{isabelle,isabellesym}
\usepackage{latexsym}
\usepackage{amssymb}
\usepackage{mathpartir}
\usepackage{url}
%\usepackage{tikz}
%\usepackage{pgfplots}

\usepackage[hidelinks]{hyperref}

% no right margin in quote:
\renewenvironment{quote}
{\list{}{}%
\item\relax}
{\endlist}

\newcommand{\noquotes}[1]{{\renewcommand{\isachardoublequote}{}\renewcommand{\isachardoublequoteopen}{}\renewcommand{\isachardoublequoteclose}{}#1}}

\isabellestyle{it}

\renewcommand{\isacharunderscore}{\_}
\renewcommand{\isacharunderscorekeyword}{\_}
\renewcommand{\isadigit}[1]{{\rm #1}}

% for uniform font size
\renewcommand{\isastyle}{\isastyleminor}

\newcommand{\eqnum}{\refstepcounter{equation}\hfill(\theequation)}

\hyphenation{Isa-belle}


\begin{document}

\title{A Deciable Fragment of Separation Logic}
\author{Florian Sextl}
\date{Technical University of Munich\\[\baselineskip] \today}
\maketitle

\begin{abstract}
  We present a formalization of the influential paper A Decidable Fragment of
  Separation Logic \cite{JoshBerdine.2004} by Berdine et al.
  This formalization follows the original paper in great detail and serves both
  as a follow up to a seminar paper \cite{seminar-paper} as well as my submission
  for the Be creative! homework challange in our Semantics of Prgramming Languages
  course.
  Another noteworthy followup of the aforemnetioned seminar paper is the proof-of-concept
  decision procedure Alice\_rs \cite{Alice-rs} which was implemented before this formalization.
\end{abstract}

\tableofcontents

\pagebreak

% sane default for proof documents
\parindent 0pt\parskip 0.5ex

\input{session.tex}

%\bibliographystyle{splncs03}
\bibliographystyle{acm}
\bibliography{root}

\end{document}
